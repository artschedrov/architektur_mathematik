\section{О чем эта книга}

Простое понятие множества можно рассматривать как основу современной математики: стая волков, гроздь ягод или стая голубей - это примеры множества вещей. В математических терминах эти вещи являются элементами рассматриваемых множеств - так сказать, их "членами". Отношения между элементами множества характеризуют его математическую структуру. И хотя математика состоит из более чем трех тысяч различных специализированных дисциплин, удивительно, что ее основное здание покоится всего на трех базовых структурах, на трех столпах, благодаря которым становится понятной вся математика: порядковая структура, алгебраическая структура и топологическая структура.
Архитектура математики - это попытка взглянуть на математику с высоты птичьего полета и описать общий знаменатель всех математических объектов и содержаний - как глобальную архитектурную надстройку и идеальное закругление одновременно.

Пьер Басё изучал математику, физику и философию, защитил докторскую диссертацию по исследованию операций и теории игр и несколько лет работал учителем в средней школе. В 1980-х годах он отвечал за планирование, контроль и логистику в многонациональной корпорации. С 1990 года работает как независимый консультант по управлению. Опубликовал множество книг, в том числе стандартный труд "Рулетка: игра на деньги" (5-е издание, Мюнхен, 2001).
Его книги "Die Welt als Roulette: Denken in Erwartungen" (rororo № 19707), "Abenteuer Mathematik: Brücken zwischen Wirklichkeit und Fiktion" (rororo № 60178) и "Die Top Ten der schönsten mathematischen Sätze" (rororo Mr 60883) были опубликованы в научной серии.
\pagebreak
