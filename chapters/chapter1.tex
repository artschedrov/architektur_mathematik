\section{Объекты мышления: множества, предметы, отношения}
\subsection{Основные объекты: Множества и их элементы}
Математикам всегда приходится иметь дело с некоторыми объектами, которые под разными углами складываются в единое целое. Такие единицы называются множествами. Согласно Георгу Кантору, создателю теории множеств, множество - это "объединение определенных, хорошо дифференцированных объектов нашего восприятия или мышления в единое целое". Объекты, обобщенные таким образом, образуют элементы множества. Это основное определение того, что сегодня известно как "наивная теория множеств".

Современные математические теории говорят о множествах, между элементами которых существуют определенные отношения, определяемые базовыми понятиями, называемыми аксиомами. Именно это разнообразие возможных отношений между элементами множества приводит к разнообразию структурированных множеств, которые являются основной темой данной книги. Можно сказать, что именно эти отношения порождают самые разнообразные структуры. С другой стороны, множества без отношений между их элементами выглядят как аморфные сущности.

Особое достижение Кантора заключается в исследовании множеств с бесконечным числом элементов. С помощью своей "теории мощностей" он создал возможности для анализа в области бесконечности. (В этой главе, однако, мы ограничимся рассмотрением некоторых основ алгебры множеств).

Канторовское определение множества оказывается очень общим и выходит за рамки традиционных математических объектов - что вполне соответствовало духу изобретателя. Его "объекты для созерцания или размышления" включают в себя числа, зонтики и телевизоры, а также такие понятия, как свобода или избирательная тайна.

Однако сама широта концептуализации оказалась губительной. Если определение множества столь широко, то необходимо допустить и такие множества, как "множество всех абстрактных понятий" или даже "множество всех множеств", то есть множества, которые содержат сами себя. Это объясняется тем, что множество всех абстрактных понятий само является абстрактным понятием, а множество всех множеств само является множеством. Британский логик и философ Бертран Рассел показал, что такие общие концептуализации приводят к противоречиям (антиномиям). На самом деле достаточно спросить, содержит ли "множество всех множеств, не содержащих себя в качестве элемента" в понимании Рассела, себя в качестве элемента. Предположение, что это так, немедленно приводит к выводу, что оно не содержит себя в качестве элемента (потому что оно содержит именно те множества, которые не содержат себя в качестве элементов). С другой стороны, если предположить, что множество Рассела не содержит себя, то оно принадлежит множествам, которые обобщены в этом (Расселовском) множестве - следовательно, оно должно содержать себя как элемент. Независимо от ответа, вопрос о том, бреет ли себя цирюльник, который бреет всех, кто не бреется сам, приводит к противоречию.

Чтобы избежать подобных формулировок и вытекающих из них противоречий, базовые понятия и операции были заданы в системе строгих определений, которая получила название "аксиоматическая теория множеств" - в отличие от "наивной теории множеств" (аксиомы теории множеств также перечислены в главе об упорядоченных множествах). Для большинства математических целей вполне достаточно интерпретировать определение Кантора таким образом, что элементы множества должны быть определены еще до того, как они будут объединены в множество. Поскольку ни одно множество не содержит всего - не существует и всеобщего множества. Более философский подход: совокупность, которая содержала бы все объекты (множества) теории, сама не могла бы быть объектом теории. Поэтому определение элементов множества не должно сначала производиться самим множеством. Тогда исключаются множества, содержащие самих себя в качестве элементов, и больше нет места для противоречий.

Чтобы определить множество практически, у нас есть два способа: Либо мы перечисляем все его элементы, либо определяем множество по характерному свойству его элементов. Конечно, явное перечисление возможно только в том случае, если множество содержит конечное число элементов.

Несколько примеров проиллюстрируют эти возможности. Обозначения множеств даны заглавными латинскими буквами.

Также принято просто писать элементы рассматриваемого множества между фигурными скобками, например:

\vspace{0.5cm}
\(A = \{1,3,5,a,b,x,c\}\)
\vspace{0.5cm}

Вы также можете использовать просторечия для определения количества с помощью подходящего (однозначного) свойства:

\vspace{0.5cm}
B = Решения квадратного уравнения \(x^2+7x-1=0\)

X = буквы слова КАНТОР
\vspace{0.5cm}

Если x - элемент множества M, то мы пишем для него \(x \in M\).

Если x не принадлежит M, это выражается через \(x \notin M\)
\vspace{0.5cm}

Вот несколько примеров бесконечных множеств, где буквы представляют собой традиционные имена множеств с фиксированным значением:

\vspace{0.5cm}
N = множество натуральных чисел \(\{1,3,5,\dots\}\)

\(N_0\) = множество N с нулевым значением \(\{0,1,2,\dots\}\)

F = \(\{3n + 5|n \in N_0\}=\{5,8,11,14,17,\dots\}\)
\vspace{0.5cm}

Читается: F равно множеству всех элементов вида 3n + 5, где n принадлежит множеству натуральных чисел (включая ноль).

\vspace{0.5cm}
Z = множество целых чисел \(\{\dots,-2,-1,0,1,2,\dots\}=\{0,-1,1,-2,2,-3,3,\dots\}\)

Q = множество рациональных чисел или дробей \(\{\frac{p}{q}|p,q \in Z,q \neq 0\}=\{\frac{p}{q}|p \in Z,q \in N\}\)

I = \(\{x|x \in Q, 0 \le x \le 1\}\)
\vspace{0.5cm}

В разговорной речи I также можно назвать множеством всех рациональных чисел x, удовлетворяющих неравенству \(0 \le x \le 1\).
Важно, чтобы оно было однозначным: должно быть ясно, является ли каждая мыслимая, возможная вещь элементом рассматриваемого множества или нет.

Используя обозначения \(S \subset T\) или \(S \subseteq T\), мы указываем, что множество S является подмножеством множества T, то есть каждый элемент \(s \in S\) также содержится в T: \(s \in T\). Эквивалентно, \(T \supset S\)(или \(T \supseteq S\)) означает, что T является надмножеством S и поэтому содержит это множество. (Я использую символы \(\subset\) и \(\subseteq\) равнозначно.) Из этого определения сразу следует, что каждое множество X содержит себя в качестве подмножества: \(X \subseteq X\) для каждого множества X.

Равенство множеств определяется следующим образом: Два множества равны тогда и только тогда, когда в них есть одинаковые элементы. Множества могут также выступать в качестве элементов других множеств, но, разумеется, всегда следует проводить точное различие между символами \(\in\) и \(\subset\) (или \(\subseteq\)); например:

\vspace{0.5cm}
\(M = \{a,b,c,\{1,2,3\}\}\) или также \(M = \{a,b,c,D\}\) с \(D=\{1,2,3\}\)
\vspace{0.5cm}

В этом случае:

\vspace{0.5cm}
\(\{1,2,3\}=D \in M\), а также \(a \in M\),

но

\(\{\{1,2,3\}\}=\{D\} \subset M\), а также \(a \subset M\).
\vspace{0.5cm}

По мере необходимости я буду вводить дополнительные обозначения.

\subsection{Много шума из ничего: Пустое множество}
Рассмотрим множество G целочисленных решений простого уравнения \(2x-1=0\).

Множество G однозначно определяется свойством его возможных элементов.
Решением уравнения является \(x=\frac{1}{2}\), но это не целочисленное решение.
Поэтому множество G не содержит ни одного элемента. Мы пишем \(G=\{\}\) или \(G=\emptyset\) и называем это множество пустым или нулевым.

Расширение понятия о множествах так же оправдано, как и введение нуля в числовой ряд.
Оно позволяет, как в нашем примере, определять множества, для которых заранее неизвестно, содержат они элементы или нет.
Таким образом, пустое множество существует, и оно однозначно.
Определение подмножества сразу же приводит к формальному утверждению: пустое множество является подмножеством любого множества.

\subsection{Диаграммы множеств и элементарные операции над множествами}
Для лучшей визуализации, множества часто представляют графически, используя диаграммы.
Однако эти диаграммы не выполняют никакой другой цели, кроме оказания мысленной помощи;
в частности, они не имеют доказательной силы. Представление множества \(M = \{a, b, c, D\}\) с \(D=\{1, 2, 3\}\) может выглядеть следующим образом:

В частности, в формальных науках очень важны процессы, заключающиеся в получении из объектов последующих (подобных) объектов: например, последующие высказывания выводятся из существующих утверждений.
Аналогично, из существующих множеств образуются последующие множества.
Часто рассматриваемые случаи получения множеств - это форма объединения и усреднения.

