\section{Объекты мышления: множества, предметы, отношения}
\subsection{Основные объекты: Множества и их элементы}
Математикам всегда приходится иметь дело с некоторыми объектами, которые под разными углами складываются в единое целое. Такие единицы называются множествами. Согласно Георгу Кантору, создателю теории множеств, множество - это "объединение определенных, хорошо дифференцированных объектов нашего восприятия или мышления в единое целое". Объекты, обобщенные таким образом, образуют элементы множества. Это основное определение того, что сегодня известно как "наивная теория множеств".

Современные математические теории говорят о множествах, между элементами которых существуют определенные отношения, определяемые базовыми понятиями, называемыми аксиомами. Именно это разнообразие возможных отношений между элементами множества приводит к разнообразию структурированных множеств, которые являются основной темой данной книги. Можно сказать, что именно эти отношения порождают самые разнообразные структуры. С другой стороны, множества без отношений между их элементами выглядят как аморфные сущности.
\subsection{Абстракция, простота и структура}
\subsection{Абстракция, простота и структура}
