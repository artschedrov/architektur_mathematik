\section{Пролог}
\subsection{Абстракция, простота и структура}
Зрители, увидевшие женский портрет Жоржа Брака, ехидно заметили, что никто не захочет встретить на улице такую изуродованную особу, на что художник ответил, что хотел написать не женщину, а картину.
В принципе, отношения между реальностью и математическим образом или моделью одинаковы: Последняя также является лишь изображением некоторых свойств реальных (или воображаемых) объектов. Все, что не относится к делу, игнорируется: например, какой сегодня день недели или что над озером туман. Моделирование произвольно, целесообразность - его единственная цель.

Для базового понимания полезно постоянно держать в голове идею абстракции в искусстве. Представьте себе математику с единой "синтезированной" точки зрения великих художников прошлого века. Пабло Пикассо, Василий Кандинский и Пауль Клее, например, видели в абстракции свободу духа - упрощение и проработку существенного. Абстракция в работах Рене Магритта и Марселя Дюшана, которые, напротив, были художниками-репрезентаторами, в гораздо большей степени заключалась в визуализации противоречий между реальным, воображаемым и изображаемым. Возможно, вы видели картину Магритта с изображением трубы, на которой написано (на первый взгляд, противоречивое) предложение Ceci n' est pas une pipe. Несомненно, из этих художников могли бы получиться отличные математики.

Преувеличенная характеристика логика и философа Бертрана Рассела, согласно которой математика - это наука, в которой вы не знаете, о чем говорите, и соответствует ли то, что вы говорите, фактам, кажется, применима к абстрактной математике.
Причиной этого отчуждения является именно абстракция. Абстракция - это лишь процесс упрощения, в ходе которого несущественное в значительной степени устраняется, абстрагируется (лат. abstrahere, отводить). Абстракция - это ни в коем случае не примитивный редукционизм, а скорее мыслительный эксперимент, идеализация, концентрация на существенном (каким бы сложным оно ни было!), упрощение, иногда до карикатуры. Это, пожалуй, самый плодотворный метод занятия наукой. Потому что с какими бы конкретными вещами ни возились: Сначала их нужно как-то осмыслить, часто в абстрактной, упрощенной форме. И никакая ясность не может быть чище, чем абстрактная ясность. Но несомненно, что мысль - это тоже категория реальности.

Конкретная реальность часто настолько сложна (не просто сложна), что мы не можем разобраться с ней, не упростив ее. Тогда мы создаем ее модель. Эта модель может все больше отдаляться от реальности и даже полностью терять с ней связь. Мы склонны называть такие модели, живущие собственной жизнью, чистой математикой - в отличие от прикладной математики, которая в первую очередь изучает проблемы, связанные с конкретной реальностью. Сама по себе абстракция не является ни хорошей, ни плохой, а лишь более или менее целесообразной. Прежде всего, ее не следует путать с арифметическими трюками.

Чистая или прикладная, абстрактная или конкретная, простая или сложная: вполне возможно, что к конкретной задаче или математическому объекту можно приписать тот или иной предикат, но я не знаю никого, кто мог бы провести четкие границы или привести убедительные доводы. Скорее, характеристики пересекаются. Конкретная проблема прикладной математики может быть очень сложной и содержать аспекты, отсылающие к чистой математике, а чистая, высшая математика может быть очень простой.

Существуют и более сложные - хотя и более конкретные - области знания, чем математика: вспомните гильдию молекулярных биологов, решающих головоломки, физиков элементарных частиц и космологов, которые ищут объединение сил природы, всеобъемлющую теорию, "формулу мира", которая должна объединить теорию относительности и квантовый мир. В принципе, любая наука, изучающая часть конкретной реальности, даже сложнее математики, потому что последняя, по сути, концентрирует свое исследование только на внутренней логике простейших объектов. Но шестеренка есть шестеренка, независимо от того, является ли она частью простой машины или одним из миллионов компонентов управляемого комплекса.

Я предполагаю, что в какой-то момент люди смогут говорить о математических объектах и вопросах, а также о политике и социальных проблемах - без формул, просто с помощью словесного обмена идеями и силы своих аргументов.

\subsection{Понятие множества как основа}
Простое понятие множества можно считать основой современной математики. Стая волков, гроздь ягод или стая голубей - это примеры множеств вещей. Эти вещи являются элементами рассматриваемых множеств, их "членами". Мысленное обобщение некоторых из этих элементов мы называем подмножеством исходной совокупности.
С одной стороны, можно рассматривать различные типы отношений между элементами или подмножествами базового множества; с другой стороны, математики также изучают множества множеств, множества множеств множеств - иногда башни ужасающей высоты и сложности; и ничего больше.
Логика, конечно, незаменима. Однако это не предмет математики, а скорее "гигиена" этого предмета - сравнимая по своей функции с грамматикой и синтаксисом, которые мы можем понимать как гигиену языка.
\subsection{Три базовые структуры для более чем трех тысяч отдельных дисциплин}
Отношения между элементами (а также между подмножествами) множества характеризуют его структуру. Хотя в математике в целом существует более трех тысяч различных специализированных дисциплин (см. Davis/Hersh, 1994), может показаться неожиданным, что ее основное здание базируется всего на трех базовых структурах, трех столпах, которые делают почти всю математику понятной: структура порядка, алгебраическая структура и топологическая структура - короче говоря, "порядки, связи и окрестности". Любое структурированное множество, каким бы сложным оно ни было, состоит из комбинации этих базовых структур. Это естественным образом приводит нас к рассмотрению множественных структур.

Несколько десятилетий назад было модно организовывать различные дисциплины математики с помощью понятия структуры. Продолжая идеи формалистической школы, основанной Давидом Гильбертом, организация Николя Бурбаки, состоящая в основном из французских математиков, попыталась создать новую структуру предмета, основанную на понятии количества и базовых структур: математика как структурное знание, находящееся между естественными и гуманитарными науками, но все еще не являющееся по-настоящему междисциплинарным, или, возможно, более уместно, если мы включаем искусство, все еще не межкультурным.

Со студенческих времен у меня остались особенно приятные воспоминания о педагогически ориентированных трудах университетских преподавателей Карла Петера Гротемейера и Герберта Мешковского, структура и содержание которых оказали мне неоценимую помощь в написании этой книги после того, как они помогли мне (помимо бурбакизма в форме почти мазохистского испытания) обрести благотворное понимание структурной математики.

Без сомнения, это математическая книга. Каждый, кто прочтет ее с пониманием, в итоге будет лучше знать, что такое математика. Однако это не сделает их лучше в математике.

Если "Математика приключений: мосты между реальностью и фантастикой" (Basieux, 1999) описывает некоторые из основных областей математики, то "Десять самых красивых математических теорем" (Basieux, 2000) и "Мир как рулетка: мышление в ожиданиях" (Basieux, 1995) посвящены конкретным положениям и особым областям. Эта книга - попытка взглянуть на математику с высоты птичьего полета и описать общий знаменатель всех математических объектов и содержания: глобальную надстройку и идеальное закругление одновременно.

На первый взгляд, ход мыслей кажется более "сферическим", чем содержание предыдущих эссе. Однако я хотел бы показать, что математическая абстракция едва ли больше, чем в некоторых "конкретных" мыслительных процессах, и что без страшной техники вычислений часто не обойтись - особенно в случае более сложных рассуждений. Речь идет не только о творческих изобретениях и открытиях в преимущественно "платоническом мире", но и о целенаправленных прозрениях, позволяющих человеческому разуму постичь простейшие структурные аспекты реальности - совершенно независимо от какой-либо материальной цели.

Данный текст построен по принципу "шаг за шагом". Я не боюсь использовать многие из самых элементарных инструментов профессии - настолько элементарных, что математик вряд ли когда-либо осознанно использует их на практике. Суть, которая часто не затрагивается или игнорируется даже в научно-популярных презентациях, для обывателя важнее, чем "как" (логические следствия и доказательства). Таким образом, увлекательная архитектура математического здания постепенно становится понятной читателю в прямом смысле этого слова. Однако "как" и "почему" также требуют от читателя неоднократного изучения реальных механизмов математического мышления, особенно мышления в важнейших областях структурной математики. В итоге они смогут ответить на вопрос, что такое математика на самом деле - и, в частности, каковы преимущества структурной математики - своим собственным способом.

Нас также поражает тот факт, что отдельные комнаты этого здания никогда не бывают полностью закончены: Совершенство достигается в бесконечном творческом процессе.
\pagebreak
